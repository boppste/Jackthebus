\subsection{Während der Woche vor der Abreise nach Deutschland}

\begin{wrapfigure}{L}{0.4\textwidth} 
  \begin{centering}
    \includegraphics[width=0.4\textwidth, height=5cm, keepaspectratio]{../Bilder/Sinsheim/47.jpg}
    \caption{Neue Isomatte für die Erhöhung des Komforts}
  \end{centering}
\end{wrapfigure} 

Nach dem Osterweekend am Walensee baute ich verschiedene Modifikationen an Jack ein.
Die grösste Sorge galt der doch recht unbequemen Multivan Sitzbank, welche alles andere als bequemer wurde.
Könnten (nicht die ersten) Alterserscheinungen sein.
Nach einer nicht sehr gründlichen Recherche, stellte sich das Problem als altbekannt heraus.
Lösungen: Mehr als genug, jedoch leider meistens mit einem massiven Umbau des Busses.
Eine einfache Lösung war jedoch mit dabei, eine selbstfüllende Isomatte.
Ist doch auf jeden Fall einen Versucht Wert und schon bald wurde eine passende bei Transa gefunden.
Das einzige Problem dabei: Der Transa in Luzern beschloss kurzerhand sein Geschäft auszubauen und hatte aus diesem Grund am Dienstag geschlossen.
Es musste wohl oder Übel am Mittwoch ein weiterer Versuch gestartet werden.
Zusätzlich wollte ich das Ladegerät für die Batterien endgültig Einbauen, da es sich am Walensee bewährt hatte.
Gerade bei dem Sauwetter wird die Batterie sonst innerhalb von zwei Tagen von der Heizung leer gesogen.
Die allergrösste Baustelle war jedoch der Tachometer, der trotz meinem Winkel einfach nicht funktionieren wollte.
Das Problem lag auch nicht nur in der herausrutschenden Achse, das wurde durch den Winkel behoben, viel mehr rutschte ein Zahnrad auf der Achse.
Dies musste unter heftigem Fluchen repariert werden.
Auch die ausgestiegene Tachobeleuchtung habe ich bei der Gelegenheit wieder funktionstüchtig gemacht. 

\subsection{20.04.2017 Sinsheim }

\begin{wrapfigure}{L}{0.2\textwidth} 
  \begin{centering}
    \includegraphics[width=0.4\textwidth, height=5cm, keepaspectratio]{../Bilder/Sinsheim/1.jpg}
    \caption{Bus ein weiteres Mal am Zumbachweg}
  \end{centering}
\end{wrapfigure} 


Kurz nach acht war der Bus fertig beladen und die Fahrt Richtung Norden ging los.
Auf dem Weg nach Basel machten sich auf einmal komische Geräusche aus dem Tacho-Gehäuse bemerkbar.
Nur nicht schon wieder dieser blöde Tachometer.
Beim Tankstopp wurde die Tachowelle grosszügig mit WD40 benebelt, das half rein gar nichts.
Musik lauter drehen war die Lösung.
Tatsächlich verschwanden die Geräusche bald darauf und sogar der Kilometerzähler drehte sich wann immer sich auch die Räder drehten.
Ein absolutes Novum.
Die Fahrt ging dann unterbrochen für einen klassischen Raststätten Burger weiter immer Richtung Sinsheim.
Vielfach war es ein Abwägen zwischen Einreihen zwischen LKWs und überholen derselben.
Trotzdem kam ich dann kurz vor nach 13:00 Uhr in Sinsheim an, die zwei Überschallflugzeuge schon gut von der Autobahn sichtbar.
Der erste Eindruck: Wahnsinn, was diese Leute hier auf die Beine oder doch eher auf die Pfeiler gestellt haben.
Bus geparkt und ab ins Abenteuer mit dem Platinum Angebot welches Sinsheim, Speyer und in beiden Orten noch das IMAX Kino beinhaltet.
Die Menge der Exponate ist schlichtweg überwältigend.
So kommt teilweise das Gefühl auf, dass Weniger mehr wäre.
Alles steht durcheinander und während dem die Fahrzeuge im schönsten Glanz erstrahlen, hängen die Flugzeuge unter einer Dicken Staubschicht unter der Decke.
Auch im Innern der Concorde ist leider schon länger nicht mehr gereinigt worden.
Hinter den schützenden Plexiglas-Abdeckungen hat sich der Staub schon länger ausgebreitet.
Trotz alle ist es grossartig die beiden Überschallpassagierflugzeuge nebeneinander betrachten zu können.
Nach dem IMAX-Besuch und einer kurzen Stärkung ging es auf die zweite Runde und kurz vor sechs Uhr machte ich mich auf den Weg Richtung Speyer wo ich direkt auf dem Caravaning Stellplatz neben dem Technik Museum übernachten wollte.
Die kurze Fahrt sollte kein Problem sein\dots{} doch leider standen da noch etliche Wagen zwischen mir und dem heutigen Reiseziel: Feierabendverkehr.
Nach gut 3-facher Fahrzeit kam ich dann in Speyer an und checkte auf dem Stellplatz neben den mächtigen Wohnmobilen ein.
Das herrliche Wetter lud zu einem Spaziergang Richtung Dom ein und direkt neben dem Stellplatz war die 747 auf den riesigen Stützen zu sehen.
Dank der Standheizung und der neuen Matratze war die Nacht eigentlich äussert angenehm. 

\begin{figure}[t]
    \centering
    \includegraphics[width=0.4\textwidth]{../Bilder/Sinsheim/3.jpg}
    \caption{Ankunft in Sinsheim}
    \label{img:Sinsheim}
\end{figure}

\begin{figure}[H]
   \centering
      %\subfloat[CAPTION]{BILDERCODE}\qquad
   \subfloat{\includegraphics [width=0.3\textwidth]{../Bilder/Sinsheim/8.jpg}}\quad
   \subfloat{\includegraphics [width=0.3\textwidth]{../Bilder/Sinsheim/9.jpg}}\quad
   \subfloat{\includegraphics [width=0.3\textwidth]{../Bilder/Sinsheim/11.jpg}}\quad
   \caption[Concorde und TU-144]{Concorde und TU-144}
\end{figure}

\newpage

\subsection{21.04.2017 Speyer }
Nach einem gemütlichen Morgen mit einer angenehmen Dusche und umfangreichen Frühstück genoss ich die wärmende Sonne auf dem Liegestuhl und las ein Buch.
Kurz vor Mittag ging dann der zweite Tag der grossen Museumstour los.
Speyer ist nicht weniger beeindruckend als Sinsheim.
Gerade die auf hohen Stützen positionierte Boeing 747 macht doch mächtig Eindruck.
Auch die Halle mit dem Russischen Space Shuttle Buran ist schön hergerichtet.
In dieser Halle findet der interessiere Besucher auch viele Hintergrund Informationen über die Exponate und zu den zahlreichen Apollo, Mercury und weiteren NASA Missionen.
Das IMAX \quotedblbase DOME\textquotedblleft{} Kino ist gewaltig auch wenn hier die Platzwahl entscheidender ist als im normalen IMAX Film Theater.
Der Stau am Vorabend hat mich von der Idee geheilt noch am selben Abend in die Schweiz zurück zu fahren, zu dem ist Speyer auf jeden Fall noch einen Besuch wert.  

\begin{figure}[H]
   \centering
      %\subfloat[CAPTION]{BILDERCODE}\qquad
   \subfloat{\includegraphics [width=0.3\textwidth]{../Bilder/Sinsheim/37.jpg}}\quad
   \subfloat{\includegraphics [width=0.3\textwidth]{../Bilder/Sinsheim/40.jpg}}\quad
   \subfloat{\includegraphics [width=0.3\textwidth]{../Bilder/Sinsheim/42.jpg}}\quad
   \caption[Campingplatz und Boeing 747]{Campingplatz und Boeing 747}
\end{figure}

\subsection{22.04.2017 Rückfahrt nach Nussbaumen }
Schon um sieben Uhr ging das Licht im Volumen-gemessen kleinsten Fahrzeug auf dem Stellplatz an.
Meine sieben Sachen wurden verstaut und schon kurz darauf ging die Rückreise los. 

\begin{figure}[b]
    \centering
    \includegraphics[width=0.7\textwidth]{../Bilder/Sinsheim/46.jpg}
    \caption{Das Städtchen Speyer}
    \label{img:Sinsheim2}
\end{figure}
