\subsection{28.05.2013}
Nach der wilden Hin -und Herfahrt zwischen Buochs und Dättwil kam ich kurz vor Mittag inklusive Bus und 80\% vom Gepäck in Dättwil an.
Der Plan war möglichst schnell nach Süden aufzubrechen und unterwegs einen Plan zu schmieden.
Um 16:00 war der Bus beladen, sich überall verabschiedet und der Verkehr hoffentlich noch erträglich.
Nicht der Verkehr, sondern mein Magen, genauer gesagt ein leichtes Hungergefühl zwang uns zur ersten Rast im Knonauer Amt.
Der kurze Unterbruch wurde zudem dazu benutzt die Vorräte im Bus aufzustocken.
Die Fahrt verlief absolut ohne Ereignisse, bis wir das Urnerland erreichten und wir auf den Föhn trafen.
Die eher grosse Fläche des Buses führte zu einem wilden Zick-Zack Kurs auf der Autobahn.
Auch dieses Hindernis wurde jedoch bezwungen und schon bald kam das nächste in Sicht.
Die schweren Wolken, welche auf der Südseite der Alpen zu sehen waren, verhiessen nichts Gutes.
Kaum aus dem langen Loch Gotthard und schon waren wir im typischen Tessiner Regen.
Kurz vor Bellinzona wurde schnell ein Halt eingelegt um den Abend zu planen.
Mittels Trip Advisor war schnell ein Hotel gefunden.
In Melide.
Taktisch geschickt gelegen um am nächsten Tag weiter nach Verona zu fahren und noch einen Zwischenhalt im Foxtown zu machen.
Das Hotel Dellago empfing uns mit einem Zimmer komplett in Pink.
Die Aussicht von der Terrasse des Hauseigenen Restaurants war wunderschön und so war die Diskussion, wo heute unser Abendmahl eingenommen werden soll schnell erledigt.
Die angebotenen Speisen wurden per Ipad präsentiert und auch sonst war alles durchdesignt.
Zur Feier des Tages gab es kein Halten mehr.
Entrecote, Spargelcremesuppe mit Lachs und Jakobsmuscheln fanden den Weg auf unsere Teller und wurden genüsslich verschlungen.
Der weitere Abend fiel relativ kurz aus, da Chantal mit heftiger Müdigkeit auf den Wein reagierte.

\subsection{29.05.2013}
Nach einer erholsamen Nacht lockte das Morgenbuffet.
Der Regen prasselte nieder, jedoch hellte das reichhaltige Buffet mit hausgemachter Marmelade unsere Gemüter auf.
Auch der Wettergott hatte ein Einsehen und liess die Regenwolken durch blaue Fetzen des Himmels ersetzen.
Bei schönstem Wetter machten wir uns auf Richtung Chiasso.
Da ja eigentlich schlechtes Wettter vorhergesagt war machten wir noch einen Stopp im Foxtown.
Etliche Kleiderständer und manche Einkaufstaschen später ging es weiter nach Verona.
Natürlich um zu shoppen...
Eine ereignisfreie Zeit später kamen wir in Verona an und parkierten am selben Ort wie beim letzten Besuch.
Der North Face Shop war schnell gefunden und nach einer gemeinsamen Glace wurde eine Zeit abgemacht, damit Chantal ihrem natürlichen Trieb nachgehen konnte und die Geschäfter unsicher machen konnte.
Nachdem die neu erworbenen Gegenstände im Bus verstaut waren, hiess der nächste Halt Gardasee.
Schon nach kurzer Fahrt kamen wir nach mehreren kurz besuchten Dorfplätzen in Bardolino an.
Als erstes versuchten wir einen Platz für die Nacht zu finden, was uns auf dem Campingplatz Continental auch gelang.
Nach dem ersten Apéro und manchen Schnappschüssen des Sonnenuntergangs, machten wir uns zu Fuss auf den Weg nach Bardolino.
Während des Abendessen kam die sich am Westufer ausbreitende schwarze Wand immer näher.
Sonnenschirme flogen durch die Luft und Gläser zerschlugen auf dem Boden.
Das Personal versuchte den Schaden zu minimieren.
Kurze Zeit  später prasselte der Regen auf die Strandpromenade.
Wir blieben glücklicherweise noch ein bisschen Sitzen und konnten nach dem Regen absolut trocken zu unserem Bus zurückkehren und friedlich einschlummern.  
\subsection{30.05.2013}
Mein Plan den Tag im Gardaland zu verbringen machte ein kurzer Halt im Nachbardorf von Bardolino zu Nichte.
Das schöne Dörfchen Lazise verleitete zu einem Halt.
Wir schlenderten durch die schönen Gassen und Plätze und auch die Zuhausegebliebenen wurden nicht vergessen und mit neuen Lederwaren eingedeckt.
Nach einem wunderbaren Bruschetta ging die Fahrt Richtung Salo weiter.
Auch dieses Mal war ein Camping schnell gefunden und die Entscheidung zwei Tage hier zu bleiben fiel uns leicht.
Die Velos wurden bereit gemacht und schon ging die Fahrt nach Salo los.
Kleinere Unwege ausgenommen fanden wir das Dörfchen problemlos und machten es uns in einer Bar am Ufer des Gardasees bequem.
Die kühne Wahl von Chantal stieg ihr sogleich in den Kopf :).
Ausnahmsweise bestellte ich eine Pizza.
Das Essen wurde nur durch das kühle Wetter getrübt.
Die Fahrt zurück zum Bus war dann  dafür umso rasanter.
Chantal zündete den Turbo und freute sich schon auf dem Weg auf den kleinen Elektro-Ofen im Bus.
Nach kurzem Lesen war auch dieser Tag schon wieder zu Ende und eine weitere kühle Nacht stand uns bevor.

\subsection{31.05.2013}
Leider besserte sich das Wetter nicht wirklich.
Das aufstehen wurde von Regengeprassel begleitet.
Ausflüge mit dem Velo waren also nicht drin.
Immerhin es sollte schon bald aufhören zu regnen, jedoch wirklich schön wurder es nicht.
Chantal konsultierte den Reiseführer und die Karten und schlug vor das Dorf ... zu besuchen.
Doch bevor es mit dem Bus auf den Weg ging, wollten wir noch den Campingplatzeignen Hügel besteigen.
Das Areal des Campings erstreckte sich viel weiter als zuvor angenommen.
Ganze Quartiere von Bungalows befanden sich auf dem Hügel.
Die Fahrt führte uns durch das schon bekannte Salo und weiter der wunderschönen Küste entlang Richtung Norden.
Kurz vor unserem Etappenziel fanden wir auch schon eine Tiefgarage, welche wir als temporäre Bleibe für Jack benutzen wollten.
Leider machte ein ziemlich tief hängender gelb-schwarz gestreifter Balken uns auf die Tatsache aufmerksam, dass diese Tiefgarage relativ tief ist.
Mit Chantals geschulten Blick wagten wir uns unter der Schranke durch, wo wir zahlreiche freie Parkplätze vorfanden.
Typisch Schweizerisch machte ich mich auf die Suche nach dem Billetautomaten.
Dieser akzeptierte zwar meine Euros, liess sich aber nicht dazu bewegen seinerseits ein Ticket auszuspucken.
Bei genauerer Betrachtung der parkierten Autos, fiel uns schnell auf das der Apparat schon länger seine Funktion verweigert.
Überall befanden sich von Hand geschriebene Zettel mit dem Hinweis, dass der Automat zwar munter Euros verspeise aber keine Gegenleistung erbringe.
Das schöne Dorf war schnell besichtigt und darum lockte uns schon bald eine Bar.
Nach einem ¿kurzen¿ Spaziergang in ...
auf dem Rückweg wollten wir noch einmal Salo einen Besuch abstatten, was sich als gar nicht so leicht erwies.
Die Parkplatzsituation glich eher Zürich.
Als wir dann endlich einen Parkplatz erspäht hatten und darauf warteten, dass der Vorbesitzer endlich seine Rostlaube rausmanövriert hat, stellte ein frecher Österreicher auf der Gegenseite ebenfalls den Blinker.
Gott sei dank blockierte der talentfreie Vorbesitzer mit seinem Wagen den aufmüpfigen Österreicher für die entscheidende Sekunde, so dass mit Schwung die Parklücke in unseren Besitz überbringen konnte.
Leider wurden ab diesem Datum keine weiteren Texte mehr geschrieben... 
